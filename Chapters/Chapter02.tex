%*****************************************
\chapter{Why night-time lights}\label{ch:whynighttimelights}
%*****************************************
%\setcounter{figure}{10}
% \NoCaseChange{Homo Sapiens}
Before proceeding with the data analysis, it is first necessary to dwell on the causal relationship between night lights and the economy. 
The first connection that needs to be made concerns the physical source of night-time lights: light bulbs that run on electricity. Apart from naturally occurring night-time lights, it should be no surprise that electricity consumption is the primary cause of human light production.
There is a vast literature studying the causal link between electricity production and GDP, but unfortunately, there are no clear conclusions and depending on the country taken as a reference or the time period, the direction of causality is different. With this thesis, I will not enter directly into this debate, but I will show how night lights can contribute. 
We have said that night lights are closely related to electricity consumption. 

While electricity consumption enters fully into the production function of a country, i.e. it tells us the aggregate of how many industrial machines are currently running, how many electric motors are switched on, or how much air conditioning is used. While electricity consumption gives us an aggregate of the energy consumed in a country, night lights give us a particular portion of it. That is, that component of the electricity consumed that produces illumination. 
The connection between the amount of light emitted and its distribution in space, as I will explain later, tells us a lot about the well-being of a country. 
Observing the lights emitted by North Korea, one is surprised by the very small amount of light emitted at night, so much so that South Korea looks like an island. Yet, assuming one can know the country's energy consumption, which is not disclosed, it is plausible that North Korea consumes far more energy than many other African states with a higher amount of night light. Although there is a total lack of studies on this, it is likely that almost all of the available energy is used in war industries or propaganda activities, resulting in an extreme contraction of consumption and the life of citizens.
Therefore, night-time lights allow us to study the actual energy used by the population and thus discriminate between the regime's publicity parades and the enormous amount of energy used to support the imperial war effort.
It is therefore plausible to think that night lights tell us more about the consumption component of the gross domestic product and the population's living standards; or, the investment part that has a close connection with the population's consumption: namely airports, theatres, cinemas or street lighting. 

Moreover, night lighting is influenced mainly by the consumption preferences of different populations. For instance, Italy, like other European countries, has more light pollution and produces more light at night.

\section{Economic connections}
Economic production is deeply involved with electricity consumption. From the elementary
national income identity:
\begin{equation}
    Y=C+I+G
\end{equation}
It is straightforward to see that each element on the right-hand side produces night-time lights. For 
instance, after income growth, individuals may decide to use the car more, maybe to reach an 
outdoor concert or visit a nearby city. These are all events that produce "light" consumption.
On the investment side, similar behaviour can be assumed. Due to increased taxation flows, 
governments may decide to build new infrastructures such as bridges, ports or railways, causing 
massive new light consumption. 
For these reasons, night-time lights can be used as a proxy for GDP, especially for its components
that most produce lights
\subsection{Roads}
Roads and highways provide valuable information on investment and consumption. The construction of new infrastructure is part of the investment component, which is an important determinant of GDP, as is the modernisation of lighting.
At the same time, road traffic tells us a lot about the movement of people and thus the consumption component. Queues and traffic tell an eloquent story about the dynamism of a country and the number of people moving at a certain time.


\begin{figure}[h!]
    \centering
    \subfloat[\centering Google Maps]{{\includegraphics[width=5cm]{images/northitaroads_maps.png} }}%
    \qquad
    \subfloat[\centering Night-time lights]{{\includegraphics[width=5cm]{images/northitaroads_night.png} }}%
    \caption{North Italy roads - A1 and A14 Highways through Bologna and Modena}%
    \label{fig:northitalyroads}
\end{figure}

\begin{figure}
    \centering
    \subfloat[\centering Google Maps]{{\includegraphics[width=5cm]{images/delhiroads_maps.png} }}%
    \qquad
    \subfloat[\centering Night-time lights]{{\includegraphics[width=5cm]{images/delhiroads_night.png} }}%
    \caption{New Delhi periphery roads}%
    \label{fig:newdelhiroads}
\end{figure}
By observing night-time lights, the lines of communication between cities or towns can be studied in depth. \autoref{fig:northitalyroads} shows an extract of the main communication line in northern Italy, namely the A1 and A14 motorways connecting the cities of Bologna and Modena with Milan. \autoref{fig:newdelhiroads}, on the other hand, shows an excerpt of the roads connecting New Delhi in India and peripheral cities.

\subsection{Cities}
Cities are the primary producers of night lights. With the study of satellite imagery, the geographical distribution of urban settlements and some of the brightest elements of the cities can be observed, such as major roads,  airports and ports.
\begin{figure}[h!]
    \centering
    \subfloat[\centering Google Maps]{{\includegraphics[width=5cm]{images/milan_maps.png} }}%
    \qquad
    \subfloat[\centering Night-time lights]{{\includegraphics[width=5cm]{images/milan_night.png} }}%
    \caption{Milan - Italy}%
    \label{fig:milan}
\end{figure}
\autoref{fig:milan} shows the city of Milan with \autoref{fig:london} shows London. They are both characterized by well known "web" style. It can be recognized some roads connecting point of interest of the cities and the main airports of both cities can be easily individuated.
\begin{figure}[h!]
    \centering
    \subfloat[\centering Google Maps]{{\includegraphics[width=5cm]{images/london_maps.png} }}%
    \qquad
    \subfloat[\centering Night-time lights]{{\includegraphics[width=5cm]{images/london_night.png} }}%
    \caption{London - United Kingdom}%
    \label{fig:london}
\end{figure}
\graffito{Note: The content of this chapter is just some dummy text.
It is not a real language.}

\subsection{Airports}
Airports are usually the brightest part of the cities. In \autoref{fig:malpensa} and in \autoref{fig:charlesdegaulle} two of Europe's busiest airports.
\begin{figure}[h!]
    \centering
    \subfloat[\centering Google Maps]{{\includegraphics[width=5cm]{images/malpensa_maps.png} }}%
    \qquad
    \subfloat[\centering Night-time lights]{{\includegraphics[width=5cm]{images/malpensa_night.png} }}%
    \caption{Malpensa Airport, Milan - Italy}%
    \label{fig:malpensa}
\end{figure}
\begin{figure}[h!]
    \centering
    \subfloat[\centering Google Maps]{{\includegraphics[width=5cm]{images/charlesdegaulle_maps.png} }}%
    \qquad
    \subfloat[\centering Night-time lights]{{\includegraphics[width=5cm]{images/charlesdegaulle_night.png} }}%
    \caption{Charles De Gaulle Airport, Paris - France}%
    \label{fig:charlesdegaulle}
\end{figure}
\subsection{Ports}
 In \autoref{fig:rotterdam}, the port of Rotterdam. One of the Europe's largest commercial ports. 
\begin{figure}[h!]
    \centering
    \subfloat[\centering Google Maps]{{\includegraphics[width=5cm]{images/Rotterdamport_maps.png} }}%
    \qquad
    \subfloat[\centering Night-time lights]{{\includegraphics[width=5cm]{images/Rotterdamport_night.png} }}%
    \caption{Malpensa Airport, Milan - Italy}%
    \label{fig:rotterdam}
\end{figure}
\subsection{Human settlements}
 \autoref{fig:nile} shows the night-time image of the Nile. This image is particularly fascinating as it shows a well-known fact in historiography, namely that populations tend to settle around water sources. Indeed, it is unsurprising that most of Egypt's gross domestic product is concentrated around its primary water source.
\begin{figure}[h!]
    \centering
    \subfloat[\centering Google Maps]{{\includegraphics[width=5cm]{images/nile_maps.png} }}%
    \qquad
    \subfloat[\centering Night-time lights]{{\includegraphics[width=5cm]{images/nile_night.png} }}%
    \caption{Nile river, Egypt}%
    \label{fig:nile}
\end{figure}

\section{Why night-time lights should be adopted} % \ensuremath{\NoCaseChange{\mathbb{ZNR}}}
The employment of night lights makes it possible to estimate economic variables under special conditions in situations where we cannot obtain data or have no guarantee of its truthfulness.
Night lights have several interesting properties that have enabled their usage in economic literature. 
\begin{description}
  \item[Exogenous nature:]
 Night-time lights' measurement errors are independent of economic variables'. This allows them to be used as an alternative measure of the economy to the standard ones. In addition, growth estimates or forecasts pass through government institutes that may have an interest in manipulating data for propaganda interests. Night-time lights observable by satellites are much more challenging to manipulate. 
 
In this regard, a popular working paper \citep{martinez2018much}, forthcoming in the Journal of Political Economy in 2022, showing how autocratic states most likely overestimated their growth estimates. Martinez argues that "Since governments themselves usually produce these estimates, they face a recurring temptation to exaggerate just how well the economy is doing".
His results show that the elasticities of night-time lights on GDP are systematically larger in the most authoritarian regimes and that the overestimation averages 35\%.
\item[Worldwide coverage:] night-time lights imagery are captured daily for every area worldwide. This makes it possible to conduct economic estimations in every area of the globe, including countries lacking statistical institutions or areas affected by natural disasters or wars. 

For instance, using data from Project Black Marble, researchers from NASA's Goddard Space Flight Center (GSFC) and the Universities Space Research Association (USRA) analysed the consequences of the Russian invasion of Ukraine on city lights.\footnote{\url{https://earthobservatory.nasa.gov/images/150002/tracking-night-lights-in-ukraine}}
Similarly \citep{li2018night}, studied the dynamics of night-time lights during the Iraqi civil war.
\item[High-frequency data:] data on night-time lights are published every day. As I will show later, for this time-span there are substantial data cleaning problems, however these images lend themselves to daily analysis. Given the long process of estimating GDP, it can be argued that estimating economic variables through night-time lights leads to high-frequency data. Unfortunately, little has been written about the applicability of economic variables estimated from night-time lights to high-frequency studies. However, also as a result of recent developments in sensor technology (VIIRS), it is possible to consider using these data for policy evaluation analyses.
\item[High-resolution data:]
As I will explain in more detail in the following section, the night-time light data is high-resolution data. Namely, we can obtain information for areas of approximately 500m x 500m.
Using these data to study economic variables allows us to reconstruct economic data and allocate information in space. For instance, it is possible to reconstruct geographic maps by calculating the GDP produced in each square kilometre of a country and thus perform zonal statistics in even very small areas. With the high-frequency nature of data, it is possible to monitor a restricted area and observe what happens over time or how the economy of the area evolves following the adoption of a certain policy.
\end{description}



%*****************************************
%*****************************************
%*****************************************
%*****************************************
%*****************************************
