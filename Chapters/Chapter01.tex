%************************************************
\chapter{Introduction}\label{ch:introduction}
%************************************************
Human activities release traces into the environment. In this regard, we often think of air or water pollution but, usually, little attention is paid to light pollution. Namely, the light emitted by human economic activities that have characterised modern economic production since the second industrial revolution.

Most human economic activities produce light. For instance, using a car to drive to work or to visit a tourist destination produces light pollution. Similarly, building a bridge, a skyscraper, or an airport increases the amount of light emitted of a country.

For many years, measuring light sources' intensity was impossible, especially with proper spatial geolocalisation. The first attempts have been made with photos taken from aeroplanes that, however, had very little quality and impossible precise localization in the space.
With the production of specialised satellite modules in the 1960s, the effective measurement of the spatial distribution of night-time lights and their reflectance intensity became possible. With later technology advances, high-frequency publication and very high image resolution have been achieved.

This thesis aims to study the relationship between economic activity and light pollution production using satellite images of the Suomi NPP and NOAA-20 satellites. In particular, I will focus on studying the relationship between night-time lights and GDP, their causal link and the many economic applications. I will show the intriguing properties that satellite imagery have in estimating economic variables, such as their almost global coverage, their exogenous nature with respect to official GDP measurement and the high frequency of the data publication.
Finally, in chapter \ref{ch:model} I will combine the official GDP measurements with night-time lights information obtained from the satellites to create a better GDP estimate with a lower measurement error.

As per past literature, night-time lights are strongly correlated with many economic variables. However, handling such data is not straightforward; the relationship between the two is not linear and changes over time. This relation depends on the characteristics of the countries, on their share of the agriculture sector and on the level of economic development in the region.

The Defence Meteorological Satellite Program (DMSP) is a US military programme that, since the 1960s, has been involved in the launch into orbit of satellites to acquire environmental earth data. The project is jointly managed by the United States Space Force and the National Oceanic and Atmospheric Administration.
For more than ten years, i.e. from the 1960s until 1972, the operation remained highly classified due to the very high technology involved. Moreover, the tense period of the Cold War certainly played an important role as the DMSP satellites collected photos twice a day for each area of the world.
As of December 1972, the mission was declassified and began to be opened to the non-military scientific community.

The DMSP-OLS satellite was equipped with sensors capable of measuring day and night visible and near-infrared light. The collected images were used to observe meteorological systems and cloud coverage. 
While recognising cloud bodies from daytime images was quite simple, doing so with night-time images entailed obvious complications. Therefore, the sensor  was equipped with a 'Photomultiplier Tube' (PMT) capable of enhancing the light captured, thus producing images of the globe of night-time lights. From the comparison of the images from different times and, thus, from the presence or absence of light pixels attesting, as mentioned, human settlements and water surfaces, it could be stated that a certain area was covered or not by clouds.

For the purposes of this thesis, I will only use the data from night-time measurements. Such images are much easier to study in economic analysis without losing the instrument's effectiveness. Namely, there is no need to analyse complex images of the earth's surface during the day when it is sufficient to analyse night-time light sources. Furthermore, night-time data do not suffer from the intense light distortions caused by the sun.

Since the night-time light monitoring system has been operating, DMSP sensors have been the leading technology adopted. However, today, the data produced by DMSP has some limitations due to outdated technology, i.e. poor resolution, six-bit quantisation, saturation on the brightest lights, lack of in-flight calibration, and production of files with a lack of spectral layers suitable for discriminating different light sources.

A significant advance in the field was achieved when the Visible Infrared Imaging Radiometer Suite (VIIRS) was launched into orbit in October 2011 by the Suomi National Polar-Orbiting Partnership (Suomi NPP). The new type of sensor mounted on the satellite has overcome many of the problems of the old generation. Compared to DMSP-OLS, night-time light data imagery obtained by NPP-VIIRS has a finer spatial resolution ($742mt \times 742mt$ vs $5km \times 5km$ at nadir) and higher radiometric resolution (quantisation of 14 bits vs 6 bits). Moreover, an onboard calibration system has been implemented to enhance the quality of NPP-VIIRS night-time data. 
Replication code for this thesis is uploaded in a GitHub repository.\footnote{\url{https://github.com/nbadino/Master-Thesis}}
%*****************************************
%*****************************************
%*****************************************
%*****************************************
%*****************************************