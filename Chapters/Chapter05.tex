\chapter{Conclusions}\label{ch:conclusions}

In this thesis, I have shown some interesting properties of night-time lights that make them a unique source of information. 
Night-time lights are a very interesting tool for economic analysis and, due to their interesting properties, they lend themselves to being used in conjunction with more standard economic tools in sub-optimal contexts. 
The advancement of remote sensing technology may lead to further improvements in economic applicability in the future. In this sense, NASA's black marble project is already producing very high-resolution tiles with a resolution of less than 30 meters. Even if they have not been made public yet, they may be used, in the future, for new economic applications.

In chapter four, I proposed a model for reconstructing an improved GDP estimate by combining information from national statistical institutes and night-time lights.
Unfortunately, the temporal availability of the data chosen for this study doesn't allow us to make a long-run analysis and study how the trend of the difference between true and official GDP changes over time. In a recent study of \citet{li2020harmonized}, a 27 years dataset of night-time lights observations was reconstructed, harmonizing the data of past with new satellite technology. However, the resulting quality is remarkably lower than the new technology standing alone. There is, thus, a trade-off between data quality and data availability.
Data on national statistical capacity is still scarce. Future work will concern the construction of an index with a longer time period so that it can be let vary with respect to time in order to get better estimates.
In this thesis, I found that when growth rates are taken into account, extreme values from greenhouses or gas flares do not heavily influence the final result. However, if interested in values in levels, it is necessary to take these phenomena into account and understand the connection with production.
With my model, I obtained results consistent with the literature concerning the possible overestimation of GDP for some countries. Additionally, the optimal combination proved to be quite sensitive to COVID-19 economics' implications.
In conclusion, future work should focus on a more precise production function of night-time lights taking into account the distribution of sectors in the economy.
